\documentclass{beamer}

\usecolortheme{crane}

\title{Development and testing of methods for drones control}
\author{Paolo Leopardi}
\institute{Università degli Studi di Perugia}
\date{\today}

\usepackage{hyperref}
\hypersetup{
	colorlinks=true,
	linkcolor=black,
	filecolor=black,      
	urlcolor=blue,
	citecolor=cyan,
}

\usepackage{acro}

%\usepackage{enumitem}


% list of acronyms
\DeclareAcronym{cpp}{
	short = CPP,
	long = Coverage Path Planning,
}

\DeclareAcronym{uav}{
	short = UAV,
	long = Unmanned Aerial Vehicle,
}

\DeclareAcronym{gcs}{
	short = GCS,
	long = Ground Control Station,
}

\DeclareAcronym{qgc}{
	short = QGC,
	long = QGroundControl,
}

\DeclareAcronym{ros}{
	short = ROS,
	long = Robot Operating System,
}

\begin{document}
	
	\frame{\titlepage}
	
	\section{Implementation}
	
	\begin{frame}
		\frametitle{Flight Stack selection}
		Autopilot selection is made by evaluating possible pros and cons which every autopilot flight stack brings with it. Three possible solution were evaluated:
		\begin{itemize}
			\item INAV \cite{inav}
			\item PX4 \cite{px4}
			\item Agilicious \cite{foehn2022agilicious}
		\end{itemize}

	Evaluation based on the following parameters:
	\begin{itemize}
		\item configuration
		\item missions definition
		\item future developments
	\end{itemize}
	\end{frame}

	\begin{frame} \frametitle{Configuration}
		\begin{description}
			\item[INAV] videos on Youtube at this \href{https://www.youtube.com/playlist?list=PLOUQ8o2_nCLkfcKsWobDLtBNIBzwlwRC8}{link}
			\item[PX4] follow sections from \href{https://docs.px4.io/main/en/assembly/}{\textit{Basic Assembly}} to \href{https://docs.px4.io/main/en/flying/}{\textit{Flying}} in the official documentation
			\item[Agilicious] no description
		\end{description}		
	\end{frame}

	\begin{frame} \frametitle{Missions definition}
		\begin{description}
			\item[INAV] provide a \ac{gcs} which is capable of define only waypoints \href{https://www.youtube.com/playlist?list=PLOUQ8o2_nCLkfcKsWobDLtBNIBzwlwRC8}{link}
			\item[PX4] typically use \ac{qgc} as \ac{gcs}\footnote{\ac{qgc} supports only PX4 and Ardupilot}, here different missions can be defined and it is worth to note that there is also survey missions which seems particularly suited with the aim of this project
			\item[Agilicious] doesn't not provide a \ac{gcs} for missions definition, but it has a module called \href{https://agilicious.readthedocs.io/en/main/modules/references.html}{\texttt{reference}} which implements different ways of generating reference trajectories
		\end{description}	
	\end{frame}

\begin{frame} \frametitle{Future developments}
	\begin{description}
		\item[INAV] no description to interface with \ac{ros} 
		\item[PX4] has a subsection dedicated to \href{https://docs.px4.io/main/en/ros/}{ROS communication with PX4}. In addiction PX4 has a MATLAB package called UAV Toolbox Support Package for PX4 Autopilots \cite{mathworkspx4}
		\item[Agilicious] has very good structure for future developments beacause you can change controller or estimator by simply modify a \texttt{yaml} file. It's not provided a way to integrate GPS measurements. An interface  for \ac{ros} called \href{https://agilicious.readthedocs.io/en/main/integration/ros.html}{\texttt{agiros}} is provided.
	\end{description}
Both PX4 and Agilicious docs propose a simulator.
\end{frame}
	
\begin{frame} \frametitle{Conclusions}
	\begin{enumerate}
		\item PX4
		\item Agilicious
		\item INAV				
	\end{enumerate}
\end{frame}
	
	\begin{frame}[allowframebreaks]
		\frametitle{References}

		\bibliographystyle{unsrt} 
		\bibliography{../report/bibliography}
	\end{frame}
	
\end{document}